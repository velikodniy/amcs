\documentclass{amcs}

\usepackage[T2A]{fontenc}
\usepackage[utf8]{inputenc}

%Полезные математические пакеты
\usepackage{amsthm}
\usepackage{amsmath}
\usepackage{amsfonts}
\usepackage{amssymb}


%%% Графика %%%
\usepackage{graphicx}

%%% Тексты программ %%%
\usepackage{listings}
\lstset{
  breaklines=true,
  numbers=left,
  numberstyle={\scriptsize},
  showstringspaces=false,
  tabsize=4,
 }

\begin{document}

\settitle{Разведение желторотика маслянистого в условиях поймы Днестра}
\setstudent{403}{Иванова Иванна Ивановна}
\setfemale
\setsupervisor{ст. преп. кафедры ПМиИ}{Великодный В. И.}
\setchair{ПМиИ}{Кафедра прикладной математики и информатики}{доц.}{Коровай А. В.}
\maketitlepage

\begin{abstract}
  Аннотация
\end{abstract}

\tableofcontents

\ssection{Введение}
Это введение.


\section{Первый раздел}
\subsection{Подраздел}
\subsubsection{Подподраздел}
Текст раздела.

\section{Второй раздел}
Текст второго раздела.

\ssection{Заключение}


\begin{thebibliography}{99}
\bibitem{b0}
  Флах П.
  Машинное обучение. Наука и искусство построения алгоритмов, которые
  извлекают знания из данных / пер. с англ. А. А. Слинкина. --- М.: ДМК Пресс,
  2015. --- 400 с.
\end{thebibliography}

% Прложение
\appendix

\section{Исходники}
\begin{lstlisting}[language=Python]
x = 5
print(range(x))
\end{lstlisting}

\end{document}

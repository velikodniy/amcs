\documentclass{amcs}

\usepackage[T2A]{fontenc}
\usepackage[utf8]{inputenc}

% Полезные математические пакеты
\usepackage{amsthm}
\usepackage{amsmath}
\usepackage{amsfonts}
\usepackage{amssymb}

% Математические шрифты
\usepackage{mathspec}
\usepackage[charter]{mathdesign}

% Специальные символы
\usepackage{gensymb}

% Шрифты документа
\usepackage{fontspec}

\defaultfontfeatures{Scale=MatchLowercase,Mapping=tex-text}
\setmainfont[
  SmallCapsFont={Charis SIL Compact},
  SmallCapsFeatures={Letters=SmallCaps},
]{Charis SIL Compact}
\setsansfont{Input Sans}
\setmonofont{Input Mono Condensed}

% Семейства шрифтов
\newfontfamily\lstfont{Input Mono Condensed}

% Графика
\usepackage{graphicx}

% Тексты программ
\usepackage{listings}
\lstset{
  basicstyle={\small\lstfont},
  breaklines=true,
  numbers=left,
  numberstyle={\scriptsize},
  showstringspaces=false,
  tabsize=4,
 }

\begin{document}
\setcourse{Спецсеминар}
\settitle{Разведение желторотика маслянистого в условиях поймы Днестра}
\setstudent{403}{Иванова Иванна Ивановна}
\setfemale
\setsupervisor{ст. преп. кафедры ПМиИ}{Великодный В. И.}
\setchair{ПМиИ}{Кафедра прикладной математики и информатики}{доц.}{Коровай А. В.}
\maketitlepage

\begin{abstract}
  Аннотация
\end{abstract}

\tableofcontents

\ssection{Введение}
Это введение.

\section{Первый раздел}
\subsection{Подраздел}
\subsubsection{Подподраздел}
Текст раздела.

\section{Второй раздел}
Текст второго раздела.

\ssection{Заключение}
Это заключение.

\begin{thebibliography}{99}
\bibitem{b0}
  Флах П.
  Машинное обучение. Наука и искусство построения алгоритмов, которые
  извлекают знания из данных / пер. с англ. А. А. Слинкина. --- М.: ДМК Пресс,
  2015. --- 400 с.
\end{thebibliography}

% Прложение
\appendix

\section{Исходники}
\begin{lstlisting}[language=Python]
x = 5
print(range(x))
\end{lstlisting}

\end{document}
